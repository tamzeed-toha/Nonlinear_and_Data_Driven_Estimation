\documentclass[12pt]{article}
\usepackage[margin=1in]{geometry}
\usepackage{amsmath,amssymb,amsthm}
\usepackage{graphicx}
\usepackage{hyperref}


\usepackage{subcaption}

\title{Phase 02}
\author{Tamzeed Elahi}
\date{\today}



\begin{document}
\maketitle

\section*{Project Description}
The problem I am interested for the class project is a system identification problem for a 2D quadrotor system. In the context of this course, the observability of this system will be evaluated; especially for estimating the input matrix which is unknown. 
\begin{align*}
    \dot{x} &= Ax + Bu + w \\
    y &= Cx + v
\end{align*} 
Here, the B matrix is unknown. \\\\
Ultimately, we would be able to estimate the B matrix using the observability matrix and consider the feasibility of a quadrotor system with additional controls (sliding/tilting) like \cite{Nemati2014}. 

\section*{Abstract}



\section*{Introduction}



\section*{Analytical model of quadrotor system}
\begin{figure}[h!]
    \begin{subfigure}[t]{0.5\textwidth}
        \centering
        \includegraphics[width=5cm]{figures/model_diagram.png}
        \caption{Quadrotor system \cite{model_diagram}}
        \label{fig:01}
    \end{subfigure}
    \hfill
    \begin{subfigure}[t]{0.5\textwidth}
        \centering
        \includegraphics[width=5cm]{figures/free_body_diagram.png}
        \caption{Free body diagram \cite{model_diagram}}
        \label{fig:02}
    \end{subfigure}
\end{figure}
% \includegraphics[width=5cm]{model_diagram.png}

The state space model of a quadrotor system can be found in various literature\cite{K2019}\cite{Schreier2012}. For the class project, I will be using a simplified 2D model (only the y and z coordinates). The dynamic equation of the system can be estimated using Newton's second law of motion.
\begin{align*}
    \ddot{y} &= -\frac{1}{m}u_1(t) \sin{\phi} \\
    \ddot{z} &= -g + \frac{1}{m}u_1(t) \cos{\phi} \\ 
    \ddot{\phi} &= \frac{1}{I_{xx}}u_2(t) \\
\end{align*}
I assumed only two inputs (force and torque) are present in the system. Their relation with the states are nonlinear though it can be linearized with $\phi \approx 0$. It would be more interesting if the inputs are not forces and torques but rotor speeds. \\\\
States are: $x = [y, z, \dot{y}, \dot{z}, \phi, \dot{\phi}]^T$ \\\\
For now the output matrix-$C$ is considered to be an identity matrix. So, the system assumed to have a GPS and IMU sensors for full state estimation.

inputs are:
\begin{align*}
    u_1(t) &= F \\
    u_2(t) &= \tau
\end{align*}
matrices in continuous form:
\begin{align*}
    A &= \begin{bmatrix}
        0 & 1 & 0 & 0 & 0 & 0 \\
        0 & 0 & 0 & 0 & 0 & 0 \\
        0 & 0 & 0 & 1 & 0 & 0 \\
        0 & 0 & 0 & 0 & 0 & 0 \\
        0 & 0 & 0 & 0 & 0 & 1 \\
        0 & 0 & 0 & 0 & 0 & 0
    \end{bmatrix} \\
    B &= \begin{bmatrix}
        0 & 0 \\
        - \frac{1}{m} \sin{\theta} & 0 \\
        0 & 0 \\
        \frac{1}{m} \cos{\theta} & 0 \\
        0 & 0 \\
        0 & \frac{1}{I_{xx}}
    \end{bmatrix} \\
\end{align*}

\section*{Results:}

\subsection*{Linear case:}

Lets imagine that the coefficient of input matrix is unknown. So, we break down the input matrix into two parts. One part is known and the other part is unknown.

\begin{align*}
    B_{n \times m} &= B^{0}_{n \times k} B^{1}_{k \times m} \\ 
    \text{so,}  \\
    \dot{x} &= Ax + B^{0} B^{1} u \\
    &= Ax + B^{0} u_{B} \\
    y &= Cx 
\end{align*}
where $B^{0}$ is unknown and $B^{1}$ is known. $B^{0}$ is a diagonal matrix (for now we assume that). $C$ and $D$ are identity matrix.


Contiuous form of the system:
\begin{align*}
    \begin{bmatrix}
        \dot{y} \\
        \dot{z} \\
        \ddot{y} \\
        \ddot{z} \\
        \dot{\phi} \\
        \ddot{\phi} \\
        \dot{\beta_1} \\
        \dot{\beta_2} \\
        \dot{\beta_3} \\
    \end{bmatrix} &=
    \begin{bmatrix}
        0 & 0 & 1 & 0 & 0 & 0 & 0 & 0 & 0 \\
        0 & 0 & 0 & 1 & 0 & 0 & 0 & 0 & 0 \\
        0 & 0 & 0 & 0 & 0 & 0 & 0 & 0 & 0 \\
        0 & 0 & 0 & 0 & 0 & 0 & 0 & 0 & 0 \\
        0 & 0 & 0 & 0 & 0 & 1 & 0 & 0 & 0 \\
        0 & 0 & 0 & 0 & 0 & 0 & 0 & 0 & 0 \\
        0 & 0 & 0 & 0 & 0 & 0 & 0 & 0 & 0 \\
        0 & 0 & 0 & 0 & 0 & 0 & 0 & 0 & 0 \\
        0 & 0 & 0 & 0 & 0 & 0 & 0 & 0 & 0 \\
    \end{bmatrix}
    \begin{bmatrix}
        y \\
        z \\
        \dot{y} \\
        \dot{z} \\
        \phi \\
        \dot{\phi} \\
        \beta_1 \\
        \beta_2 \\
        \beta_3 \\
    \end{bmatrix} +
    \begin{bmatrix}
        0 & 0 \\
        \beta_1 & 0 \\
        0 & 0 \\
        \beta_2 & 0 \\
        0 & 0 \\
        0 & \beta_3 \\
        0 & 0 \\
        0 & 0 \\
        0 & 0 \\
    \end{bmatrix}
    \begin{bmatrix}
        F \\
        \tau \\
    \end{bmatrix} + w\\ \\
    &= 
    \begin{bmatrix}
        0 & 0 & 1 & 0 & 0 & 0 & 0 & 0 & 0 \\
        0 & 0 & 0 & 1 & 0 & 0 & 0 & 0 & 0 \\
        0 & 0 & 0 & 0 & 0 & 0 & 0 & 0 & 0 \\
        0 & 0 & 0 & 0 & 0 & 0 & 0 & 0 & 0 \\
        0 & 0 & 0 & 0 & 0 & 1 & 0 & 0 & 0 \\
        0 & 0 & 0 & 0 & 0 & 0 & 0 & 0 & 0 \\
        0 & 0 & 0 & 0 & 0 & 0 & 0 & 0 & 0 \\
        0 & 0 & 0 & 0 & 0 & 0 & 0 & 0 & 0 \\
        0 & 0 & 0 & 0 & 0 & 0 & 0 & 0 & 0 \\
    \end{bmatrix}
    \begin{bmatrix}
        y \\
        z \\
        \dot{y} \\
        \dot{z} \\
        \phi \\
        \dot{\phi} \\
        \beta_1 \\
        \beta_2 \\
        \beta_3 \\
    \end{bmatrix} +
    \lambda^T 
    \begin{bmatrix}
        0 & 0 \\
        \beta_1 & 0 \\
        0 & 0 \\
        \beta_2 & 0 \\
        0 & 0 \\
        0 & \beta_3 \\
        0 & 0 \\
        0 & 0 \\
        0 & 0 \\
    \end{bmatrix}
    \begin{bmatrix}
        F \\
        \tau \\
    \end{bmatrix} + w\\
\end{align*}

here $\lambda^T$ is the unknown input matrix. \\\\
if the inputs are not linearly independent, then $\Lambda$ would be a matrix.

$$\dot{x} = Ax + \Lambda B u + w$$

now the measurement equation would be:
\begin{align*}
    y &= Cx + v \\
    &= 
    \begin{bmatrix}
        1 & 0 & 0 & 0 & 0 & 0 & 0  & 0 & 0  \\
        0 & 1 & 0 & 0 & 0 & 0 & 0  & 0 & 0  \\
        0 & 0 & 1 & 0 & 0 & 0 & 0  & 0 & 0  \\
        0 & 0 & 0 & 1 & 0 & 0 & 0  & 0 & 0  \\
        0 & 0 & 0 & 0 & 1 & 0 & 0  & 0 & 0  \\
        0 & 0 & 0 & 0 & 0 & 1 & 0  & 0 & 0  \\
    \end{bmatrix}
    \begin{bmatrix}
        y \\
        z \\
        \dot{y} \\
        \dot{z} \\
        \phi \\
        \dot{\phi} \\
        \beta_1 \\
        \beta_2 \\
        \beta_3 \\
    \end{bmatrix} + v
\end{align*}

In the above equation, we can see the parameters dont really have any direct or indirect measurement. So, the system is not observable and also not controllable. Lets add some more sensors to the system. 

\section*{Observability}
The observability of the system can be evaluated using the observability matrix. The system is observable if the rank of the observability matrix is equal to the number of states. The observability matrix is given by:
\begin{align*}
    O = \begin{bmatrix}
        C \\
        CA \\
        CA^2 \\
        \vdots \\
        CA^{n-1}
    \end{bmatrix}
\end{align*}





\pagebreak
\bibliographystyle{plain}
\bibliography{phase0}









\end{document}